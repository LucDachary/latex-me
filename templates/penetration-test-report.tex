\documentclass[a4paper, 12pt]{report}
\usepackage[T1]{fontenc}
\usepackage[utf8]{inputenc}
\usepackage[english]{babel}
\usepackage{lmodern}
\usepackage{microtype}
\usepackage{numprint}  % pour \numprint{22950}
\usepackage{graphicx}  % To include pictures.
\usepackage{tabularx}
\usepackage{booktabs} % For prettier tables
% https://www.overleaf.com/learn/latex/Using_colours_in_LaTeX#Accessing_additional_named_colours
\usepackage[dvipsnames]{xcolor}
\usepackage{listings}
\include{json-lang}
\usepackage{csquotes}
\usepackage{float}
\usepackage{ragged2e} % for "justify" environment
\usepackage{multirow} % For \multirow{5}{*}{TCP}

\definecolor{backcolour}{gray}{0.95}
\definecolor{TmuxBar}{RGB}{108, 196, 179}
\lstdefinestyle{mystyle}{
    backgroundcolor=\color{backcolour},
    commentstyle=\color{OliveGreen},
    % keywordstyle=\color{magenta},
    % numberstyle=\tiny\color{codegray},
    stringstyle=\color{Mahogany},
    basicstyle=\ttfamily,
    breakatwhitespace=false,
    breaklines=true,
    captionpos=b,
    keepspaces=true,
    numbers=left,
    numbersep=12pt,
    numberstyle=\small,
    showspaces=false,
    showstringspaces=false,
    showtabs=false,
    tabsize=2,
	frame=shadowbox,
	rulesepcolor=\color{TmuxBar},
}
\usepackage{newverbs} %  for \collectvert
% Make \verb|...| contents show.
\renewcommand{\verb}{\collectverb{\color{black}\colorbox{TmuxBar}}}

\lstset{style=mystyle}

% -1 part (books and reports)
% 0 chapter (books and reports)
% 1 section
% 2 subsection
% 3 subsubsection
% 4 paragraph
% 5 subparagraph
% \setcounter{tocdepth}{3}
% \setcounter{secnumdepth}{3}

% https://www.overleaf.com/learn/latex/Hyperlinks
\usepackage{hyperref}
\hypersetup{
	colorlinks=true,
	linkcolor=TealBlue, % ToC and \ref{}
	% filecolor=magenta,
	urlcolor=Cerulean, % \href{}
	pdftitle={OSCP Penetration Test Report},
	pdfpagemode=FullScreen,
	breaklinks=true,
}

\renewcommand{\familydefault}{lmss}
\usepackage{fontspec}
\setmonofont{JetBrainsMono Nerd Font}[
Contextuals = Alternate,
% Ligatures = TeX,
Scale=0.8,
]

% Default \underline encloses its argument in a horizontal box, which does not allow linebreaks.
% Overriding \underline with \ul from package "soul".
\usepackage{soul}
\renewcommand{\underline}[1]{\ul{#1}}

% https://stackoverflow.com/a/123184
\usepackage{calc}
\newlength{\imgwidth}
\newcommand\scalegraphics[1]{%
    \settowidth{\imgwidth}{\includegraphics{#1}}%
    \setlength{\imgwidth}{\minof{\imgwidth}{\textwidth}}%
    \includegraphics[width=\imgwidth]{#1}%
}

% Useful symbols:
% TM \texttrademark{}

\title{TODO}
\author{TODO}

\begin{document}
\maketitle

\tableofcontents

\chapter{Executive Summary}
\section{Overview}
(This section is for senior management.)

Outline the scope of the engagement.

Write what was tested and what was dropped.

Include timing issues.

Mention the time frame:
	* length of time spent on testing;
	* dates;
	* (potentially) testing hours.

Refer to the RoEs and reference the referee report if a referee was part of
the testing team.
If DOS was allowed or Social Engineering encouraged: mention it here.
Mention any specific testing methodology.

Include support infrastructure and accounts, plus include any account that we
created.

Targets were:

\begin{tabular}{ll}
	\toprule
	Target Name & Target IP Address \\
	\midrule
	 MS01 & 192.168.191.157\\
	\bottomrule
\end{tabular}

\section{High-Level Results}
High-level overview of each step of the engagement.

Establish severity, context and worst-case scenario for the key findings from
the testing.

(No code blocks, no screenshots here.)

Make not of any trend.

Mention what the client did well (e.g.\\: proper hardering).

Conclude with an engagement wrap-up.

\chapter{Complete Report}
\section{Testing Environment Considerations}
% Here starts the full report

Detail any issues that affected the testing. Be transparent and professional.

\section{Technical Summary}
List of all the key findings, this time for a technical audience like
security architects.

Group issues into common areas.

Terminate with a risk heat map based on vulnerability adjusted as appropriate
to the client's context.

\section{Technical Findings and Remediation}
Technical section. Include the full technical details.

Often presented as a table. Write an attack narrative instead if pertinent.

\begin{tabular}{llp{0.3\textwidth}p{0.3\textwidth}}
	\toprule
	Ref. & Risk & Issue Description and Implications & Recommendations \\
	\midrule
	1 & H & Privilege management is inadequate,\ldots Some reference \ref{lst:bash-one} & All accounts should have passwords that are enforced by a strict policy.~[\ldots] \\
	\bottomrule
\end{tabular}

\chapter{Appendices}
Things that do not fit anywhere else.

Things that are too lenghty or too detailed to include inline (e.g.\\: long lists of compromised users, large proof-of-concept code blocks).

\label{listing:bash1}
\begin{lstlisting}[caption=Some long listing.,language=sh, label=lst:bash-one]
Some
long
list
\end{lstlisting}

End with references, from the most authoritative sources.

\end{document}
