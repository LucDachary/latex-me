\documentclass[NF]{scrlttr2}
% \documentclass[NF, 17pt]{scrlttr2}
% \documentclass[a5paper, 14pt]{extarticle}
% \documentclass[NF,firstfoot=false,enlargefirstpage]{scrlttr2}
\usepackage[T1]{fontenc}
\usepackage[utf8]{inputenc}
\usepackage[french]{babel}
\usepackage{lmodern}
\usepackage{microtype}
\usepackage{numprint}  % pour \numprint{22950}
\usepackage[gen]{eurosym}  % pour \euro{}
\usepackage{phonenumbers} % pour \phonenumber[country=FR]{06xxxxxxxx}

\begin{document}

\setkomavar{fromname}{TODO}
\setkomavar{fromaddress}{TODO}
\setkomavar{signature}{TODO}
\setkomavar{date}{TODO}
% See variables here: https://www.dickimaw-books.com/latex/admin/html/scrlttr2.shtml#tab:scrlttr2vars
% \setkomavar{yourref}{TODO}
% \setkomavar{customer}{TODO}

\setkomavar{subject}{TODO}

\KOMAoptions{
    foldmarks=vpmBT,
    backaddress=false,
}

\begin{letter}{
  TODO destinataire}

% See https://www.laposte.fr/courriers-colis/conseils-pratiques/rediger-une-lettre-formelle-quelles-formules-de-politesse
\opening{Madame TODO,}

TODO contenu


\closing{TODO}

\end{letter}
\end{document}
